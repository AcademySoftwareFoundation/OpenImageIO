\chapter{Image I/O API}
\label{chap:imageioapi}
\index{Image I/O API|(}


\section{Image Format Spec}
\indexapi{ImageIOFormatSpec}

An {\kw ImageIOFormatSpec} is a structure that describes the complete
format specification of a single image.  It contains:

\begin{itemize}
\item The image resolution (number of pixels).
\item The origin, if it is not located beginning at pixel (0,0).
\item The full size of any ``surrounding'' image, if this represents a crop
  window.
\item Whether the image is organized into \emph{tiles}, and if so, the
  tile size.
\item The \emph{data format} of the pixel values (e.g., float, 8-bit
  integer, etc.).
\item The number of color channels in the image (e.g., 3 for RGB
  images).
\item Which channel, if any, represents \emph{alpha} and \emph{depth}.
\item Any presumed gamma correction already applied to the pixel values.
\item Quantization parameters describing how floating point values
  should be converted to integers (if cases where users pass real values
  but integer values are stored in the file).  This is used only when
  writing images, not when reading them.
\item A user-extensible (and format-extensible) list of any other
  arbitrarily-named and -typed data that may help describe the image or
  its disk representation.
\end{itemize}

\subsection{{\kw ImageIOFormatSpec} Data Members}

The {\kw ImageIOFormatSpec} contains data fields for the values that are
required to describe nearly any image, and an extensible list of
arbitrary attributes that can hold metadata that may be user-defined or
specific to individual file formats.  Here are the hard-coded data
fields:

\apiitem{int width, height, depth}
The size of the data of this image, i.e., the number pixels in each
dimension.  If {\kw depth} is 0 or 1, it indicates a 2D image, but if
{\kw depth} is greater than 1, it's actually a 3D ``volumetric'' image.
\apiend

\apiitem{int x, y, z}
The \emph{origin} of the image.  These default to (0,0,0), but setting
them differently indicates that this image is actually a \emph{crop window}
of a larger surrounding image.
\apiend

\apiitem{int full_width, full_height, full_depth}
The size of the full surrounding image, if this image is a crop window.
The default is (0,0,0), indicating that this is the ``full'' image data
(setting these values to {\kw width}, {\kw height}, and {\kw depth},
respectively, has the same meaning).
\apiend

\apiitem{int tile_width, tile_height, tile_depth}
If nonzero, indicates that the image is stored on disk organized into
rectangular \emph{tiles} of the given dimension.  The default of 
(0,0,0) indicates that the image is stored in scanline order, rather
than as tiles.
\apiend

\apiitem{ParamBaseType format}
Indicates the format of the pixel data values themselves, as a 
{\kw ParamBaseType} (see \ref{ParamBaseType}).  Typical values would be
{\kw PT_UINT8} for 8-bit unsigned values, {\kw PT_FLOAT} for 32-bit
floating-point values, etc.

\noindent NOTE: Currently, the implementation of OpenImageIO requires
all channels to have the same data format.
\apiend

\apiitem{int nchannels}
The number of \emph{channels} (color values) present in each pixel of
the image.  For example, an RGB image has 3 channels.
\apiend

\apiitem{std:vector<std::string> channelnames}
The names of each channel, in order.  Typically this will be "R", "G",
"B", "A" (alpha), "Z" (depth), or other arbitrary names.
\apiend

\apiitem{int alpha_channel}
The index of the channel that respresents \emph{alpha} (pixel coverage
and/or transparency).  It defaults to -1 if no alpha channel is present,
or if it is not know which channel represents alpha.
\apiend

\apiitem{int z_channel}
The index of the channel that respresents \emph{z} or \emph{depth} (from
the camera).  It defaults to -1 if no depth channel is present, or if it
is not know which channel represents depth.
\apiend

\apiitem{LinearitySpec nonlinear}
Describes the type of nonlinearity, if any, that is present in the
mapping of pixel values to real-world units.  {\kw LinearitySpec} is
an enumerated type that may take on the following values:
\begin{itemize}
\item[] 
\item {\kw Linear} (the default) indicates that pixel values map
  linearly.
\item {\kw GammaCorrected} indicates that the color pixel values have
  already been gamma corrected, using the exponent given by the {\kw
    gamma} field.  (It is still assumed that non-color values, such as
  alpha and depth, are linear.)
\item {\kw sRGB} indicates that color values are encoded using the sRGB
  mapping.  (It is still assumed that non-color values are linear.)
\end{itemize}
\apiend

\apiitem{float gamma}
The gamma exponent, if the pixel values in the image have already been
gamma corrected (indicated by {\kw nonlinear} having a value of {\kw
GammaCorrected}).  The default of 1.0 indicates that no gamma
correction has been applied.
\apiend

\apiitem{int quant_black, quant_white, quant_min, quant_max;\\
  float quant_dither}
Describes the \emph{quantization}, or mapping between real
(floating-point) values and the stored integer values.
FIXME - describe this better.
\apiend

\apiitem{std::vector<ImageIOParameter> extra_attribs}
A list of arbitrarily-named and arbitrarily-typed additional attributes
of the image, for any metadata not described by the hard-coded fields
described above.  This list may be manipulated with the {\kw
attribute()} and {\kw find_attribute()} methods.
\apiend

\subsection{{\kw ImageIOFormatSpec} member functions}

\noindent {\kw ImageIOFormatSpec} contains the following methods that
manipulate format specs or compute useful information about images given
their format spec:

\apiitem{ImageIOFormatSpec (int xres, int yres, int nchans, \\
   \bigspc ParamBaseType fmt = PT_UINT8)}
Constructs an ImageIOFormatSpec with the $x$ and $y$ resolution, number
of channels, and pixel data format.

All other fields are set to the obvious defaults -- the image is an
ordinary 2D image (not a volume), the image is not offset or a crop of a
bigger image, the image is scanline-oriented (not tiled), channel names
are ``R'', ``G'', ``B,'' and ``A'' (up to and including 4 channels,
beyond that they are named ``channel \emph{n}''), the fourth channel (if
it exists) is assumed to be alpha, values are assumed to be linear, and
quantizaiton (if \emph{fmt} describes an integer type) is done in
such a way that the maximum positive integer range maps to (0.0, 1.0).
\apiend

\apiitem{void set_format (ParamBaseType fmt)}
Sets the format as described, and also sets all quantization parameters
to the default for that data type (maps the maximum positive integer
range to (0.0, 1.0)).
\apiend

\apiitem{static ParamBaseType format_from_quantize (int quant_black, int quant_white,\\
\bigspc \bigspc                          int quant_min, int quant_max)}
Utility function that, given quantization parameters, returns a data
type that may be used without unacceptable loss of significant bits.
% FIXME - elaborate?
\apiend

\apiitem{size_t channel_bytes ()}
Returns the number of bytes comprising each channel of each pixel (i.e.,
the size of a single value of the type described by the {\kw format} field).
\apiend

\apiitem{size_t pixel_bytes ()}
Returns the number of bytes comprising each pixel (i.e. the number of
channels multiplied by the channel size).
\apiend

\apiitem{size_t scanline_btes ()}
Returns the number of bytes comprising each scanline (i.e. {\kw width} pixels).
\apiend

\apiitem{size_t tile_bytes ()}
Returns the number of bytes comprising an image tile (if it's a tiled image).
\apiend

\apiitem{size_t image_bytes ()}
Returns the number of bytes comprising an image of these dimensions.
\apiend

% FIXME - document auto_stride() ?

\apiitem{void attribute (const std::string \&name, ParamBaseType type, \\
\bigspc int nvalues, const void *value)}
Add a metadata attribute to {\kw extra_attribs}, with the given name,
data type, and number of values.  The {\kw value} pointer specifies
the address of the data to be copied.
\apiend

\apiitem{void attribute (const std::string \&name, unsigned int value)\\
    void attribute (const std::string \&name, int value)\\
    void attribute (const std::string \&name, float value)\\
    void attribute (const std::string \&name, const char *value)\\
    void attribute (const std::string \&name, const std::string \&value)}
Shortcuts for passing attributes comprised of a single integer,
floating-point value, or string.
\apiend

\apiitem{ImageIOParameter * find_attribute (const std::string \&name,\\
\bigspc\bigspc                              bool casesensitive=false)}
Searches {\kw extra_attribs} for an attribute matching {\kw name}
(exactly, if {\kw casesensitive} is true, otherwise in a
case-insensitive manner) and returns the pointer to that attribute
record.
\apiend



\section{Image Output}

\apiitem{static ImageOutput * create (const char *filename, \\
\bigspc\bigspc const char *plugin_searchpath=NULL)}

Create an {\kw ImageOutput} that can be used to write an image file.
The type of image file (and hence, the particular subclass of {\kw
  ImageOutput} returned, and the plugin that contains its methods) is
inferred from the extension of the file name.  The {\kw
  plugin_searchpath} parameter is a colon-separated list of directories
to search for ImageIO plugin DSO/DLL's.

\apiend

\apiitem{const char *format_name ()}
Returns the canonical name of the format that this {\kw ImageOutput}
instance is capable of writing.
\apiend

\apiitem{bool supports (const char *feature)}
Given the name of a \emph{feature}, tells if this {\kw ImageOutput} 
instance supports that feature.  The following features are recognized
by this query:
\begin{description}
\item[\spc] \spc 
\item[\rm \qkw{tiles}] Is this plugin able to write tiled images?
\item[\rm \qkw{rectangles}] Can this plugin accept arbitrary rectangular
  pixel regions (via {\kw write_rectangle()})?  False indicates that
  pixels must be transmitted via {\kw write_scanline()} (if
  scanline-oriented) or {\kw write_tile} (if tile-oriented, and only if
  {\kw supports("tiles")} returns true).
\item[\rm \qkw{random_access}] May tiles or scanlines be written in any
  order?  False indicates that they must be in successive order.
\item[\rm \qkw{multiimage}] Does this format support multiple images
  within a single file?
\item[\rm \qkw{volumes}] Does this format support ``3D'' pixel arrays
  (a.k.a.\ volume images)?
\item[\rm \qkw{rewrite}] Does this plugin allow the same scanline or
  tile to be sent more than once?  Generally this is true for plugins
  that implement some sort of interactive display, rather than a saved
  image file.
\item[\rm \qkw{empty}] Does this plugin support passing a NULL data
  pointer to the various {\kw write_*} routines to indicate that the
  entire data block is composed of pixels with value zero.  Plugins
  that support this achieve a speedup when passing blank scanlines or
  tiles (since no actual data needs to be transmitted or converted).
\end{description}

\noindent This list of queries may be extended in future releases.
Since this can be done simply by recognizing new query strings, and does
not require any new API entry points, addition of support for new
queries does not break ``link compatibility'' with previously-compiled
plugins.
\apiend

\apiitem{bool open (const char *name, const ImageIOFormatSpec \&newspec,
                       bool append=false)}

Open the file with given {\kw name}, with resolution, and other format
data as given in {\kw newspec}.  This function returns {\kw true} for
success, {\kw false} for failure.  Note that it is legal to call {\kw
open()} multiple times on the same file without a call to {\kw
close()}, if it supports multiimage and the append flag is {\kw true}
-- this is interpreted as appending images (such as for MIP-maps).

\apiend

\apiitem{const ImageIOFormatSpec \& spec ()}
Returns the spec internally associated with this currently open
{\kw ImageOutput}.
\apiend

\apiitem{bool close()}
Closes the currently open file associated with this {\kw ImageOutput}.
\apiend

\apiitem{bool write_scanline (int y, int z, ParamBaseType format,
     const void *data, \\
\bigspc stride_t xstride=AutoStride)}

Write a full scanline that includes pixels $(*,y,z)$.  For 2D non-volume
images, $z$ is ignored.  The {\kw xstride} value gives the distance
between successive pixels (in bytes).  Strides set to the special value
{\kw AutoStride} imply contiguous data, i.e., \\
\spc {\kw xstride} $=$ {\kw spec.nchannels*ParamBaseTypeSize(format)} \\
This method automatically converts the data from the specified {\kw
  format} to the actual output format of the file.  Return {\kw true}
for success, {\kw false} for failure.  It is a failure to call {\kw
  write_scanline()} with an out-of-order scanline if this format driver
does not support random access.

\apiend

\apiitem{bool write_tile (int x, int y, int z, ParamBaseType format,
                             const void *data, \\ \bigspc stride_t xstride=AutoStride,
                             stride_t ystride=AutoStride, \\ \bigspc stride_t zstride=AutoStride)}

Write the tile with $(x,y,z)$ as the upper left corner.  For 2D
non-volume images, $z$ is ignored.  The three stride values give the
distance (in bytes) between successive pixels, scanlines, and volumetric
slices, respectively.  Strides set to the special value {\kw AutoStride}
imply contiguous data, i.e., \\
\spc {\kw xstride} $=$ {\kw spec.nchannels*ParamBaseTypeSize(format)} \\
\spc {\kw ystride} $=$ {\kw xstride*spec.tile_width} \\
\spc {\kw zstride} $=$ {\kw ystride*spec.tile_height} \\
This method automatically converts the
data from the specified {\kw format} to the actual output format of the
fil.  Return {\kw true} for success, {\kw false} for failure.  It is a
failure to call {\kw write_tile()} with an out-of-order tile if this
format driver does not support random access.

\apiend

\apiitem{bool write_rectangle ({\small int xmin, int xmax, int ymin, int ymax,
                                  int zmin, int zmax,} \\ \bigspc ParamBaseType format,
                                  const void *data, \\ \bigspc stride_t xstride=AutoStride,
                                  stride_t ystride=AutoStride, \\
                                  \bigspc stride_t zstride=AutoStride)}

Write pixels whose $x$ coords range over {\kw xmin}...{\kw xmax} (inclusive), $y$
coords over {\kw ymin}...{\kw ymax}, and $z$ coords over {\kw
  zmin}....{\kw zmax}.  The
three stride values give the distance (in bytes) between
successive pixels, scanlines, and volumetric slices,
respectively.  Strides set to the special value {\kw AutoStride}
imply contiguous data, i.e.,\\
\spc {\kw xstride} $=$ {\kw spec.nchannels*ParamBaseTypeSize(format)} \\
\spc {\kw ystride} $=$ {\kw xstride*(xmax-xmin+1)} \\
\spc {\kw zstride} $=$ {\kw ystride*(ymax-ymin+1)}\\
This method automatically converts the data from the specified 
{\kw format} to the actual output format of the fil.  Return {\kw true}
for success, {\kw false} for failure.  It is a failure to call 
{\kw write_rectangle} for a format plugin that does not return true for
{\kw supports("rectangles")}.

\apiend

\apiitem{bool write_image (ParamBaseType format, const void *data, \\
                              \bigspc stride_t xstride=AutoStride, stride_t ystride=AutoStride,
                              \\ \bigspc stride_t zstride=AutoStride, \\
                              \bigspc ProgressCallback progress_callback=NULL,\\
                              \bigspc void *progress_callback_data=NULL)}

Write the entire image of {\kw spec.width} $\times$ {\kw spec.height}
$\times$ {\kw spec.depth}
pixels, with the given strides and in the desired format.
Strides set to the special value {\kw AutoStride} imply contiguous data, i.e.,
\spc {\kw xstride} $=$ {\kw spec.nchannels * ParamBaseTypeSize(format)} \\
\spc {\kw ystride} $=$ {\kw xstride * spec.width} \\
\spc {\kw zstride} $=$ {\kw ystride * spec.height}\\
The function will internally either call {\kw write_scanline} or 
{\kw write_tile}, depending on whether the file is scanline- or
tile-oriented.

Because this may be an expensive operation, a callback may be passed.
Periodically, it will be called as follows:\\
\begin{code}
  progress_callback (progress_callback_data, float done)
\end{code}
\noindent where \emph{done} gives the portion of the image that has
been written (between 0.0 and 1.0).

\apiend

\apiitem{int send_to_output (const char *format, ...)}
General message passing between client and image output server
\apiend

\apiitem{int send_to_client (const char *format, ...)}
General message passing between client and image output server
\apiend

\apiitem{std::string error_message ()}
Returns the current error string describing what went wrong if
any of the public methods returned {\kw false} indicating an error.
(Hopefully the implementation plugin called {\kw error()} with a
helpful error message.)
\apiend


\section{Image Input}

\apiitem{ImageInput *create (const char *filename, 
                               const char *plugin_searchpath)}
Create and return an {\kw ImageInput} implementation that is able
to read the given file.  The {\kw plugin_searchpath} parameter is a
colon-separated list of directories to search for ImageIO plugin
DSO/DLL's (not a searchpath for the image itself!).  This will
actually just try every ImageIO plugin it can locate, until it
finds one that's able to open the file without error.  This just
creates the {\kw ImageInput}, it does not open the file.
\apiend

\apiitem{const char *format_name (void) const}
Return the name of the format implemented by this class.
\apiend

\apiitem{bool open (const char *name, ImageIOFormatSpec \&newspec)}
Opens the file with given name.  Various file attributes are put in
{\kw newspec} and a copy is also saved internally to the
{\kw ImageInput} (retrievable via {\kw spec()}.  From examining
{\kw newspec} or {\kw spec()}, you can discern the resolution, if it's
tiled, number of channels, native data format, and other metadata about
the image.  Return {\kw true} if the file was found and opened okay,
otherwise {\kw false}.
\apiend

\apiitem {const ImageIOFormatSpec \&spec (void) const}
Returns a reference to the image format specification of the
current subimage.  Note that the contents of the spec are
invalid before {\kw open()} or after {\kw close()}.
\apiend

\apiitem{bool close ()}
Closes an open image.
\apiend


\apiitem{int current_subimage (void) const}
Returns the index of the subimage that is currently being read.
The first subimage (or the only subimage, if there is just one) is
number 0.
\apiend

\begin{comment}

\apiitem{bool seek_subimage (int index, ImageIOFormatSpec \&newspec)}
Seek to the given subimage.  Return true on success, false on
failure (including that there is not a subimage with that
index).  The new subimage's vital statistics are put in newspec
(and also saved in this->spec).  The reader is expected to give
the appearance of random access to subimages -- in other words,
if it can't randomly seek to the given subimage, it should
transparently close, reopen, and sequentially read through prior
subimages.
\apiend

\apiitem{bool read_scanline (int y, int z, ParamBaseType format, void *data,
                                stride_t xstride=AutoStride)}
Read the scanline that includes pixels (*,y,z) into data,
converting if necessary from the native data format of the file
into the 'format' specified (z==0 for non-volume images).  The
stride value gives the data spacing of adjacent pixels (in
bytes).  Strides set to AutoStride imply 'contiguous' data (i.e.
xstride==spec.nchannels*ParamBaseTypeSize(format)).  The reader
is expected to give the appearance of random access -- in other
words, if it can't randomly seek to the given scanline, it
should transparently close, reopen, and sequentially read
through prior scanlines.  The base ImageInput class has a
default implementation that calls read_native_scanline and then
does appropriate format conversion, so there's no reason for
each format plugin to override this method.
\apiend


\apiitem{bool read_scanline (int y, int z, float *data)}
Simple read_scanline reads to contiguous float pixels.
\apiend

\apiitem{bool read_tile (int x, int y, int z, ParamBaseType format,
                            void *data, stride_t xstride=AutoStride,
                            stride_t ystride=AutoStride, stride_t zstride=AutoStride)}
Read the tile that includes pixels (*,y,z) into data, converting
if necessary from the native data format of the file into the
'format' specified.  (z==0 for non-volume images.)  The stride
values give the data spacing of adjacent pixels, scanlines, and
volumetric slices (measured in bytes).  Strides set to
AutoStride imply 'contiguous' data (i.e.  xstride ==
spec.nchannels*ParamBaseTypeSize(format), ystride ==
xstride*spec.tile_width, zstride == ystride*spec.tile_height).
The reader is expected to give the appearance of random access
-- in other words, if it can't randomly seek to the given tile,
it should transparently close, reopen, and sequentially read
through prior tiles.  The base ImageInput class has a default
implementation that calls read_native_tile and then does
appropriate format conversion, so there's no reason for each
format plugin to override this method.
\apiend


\apiitem{bool read_tile (int x, int y, int z, float *data)}
Simple read_tile reads to contiguous float pixels.
\apiend

\apiitem{bool read_image (ParamBaseType format, void *data,
                             stride_t xstride=AutoStride, stride_t ystride=AutoStride,
                             stride_t zstride=AutoStride,
                             ProgressCallback progress_callback=NULL,
                             void *progress_callback_data=NULL)}
Read the entire image of spec.width x spec.height x spec.depth
pixels into data (which must already be sized large enough for
the entire image) with the given strides and in the desired
format.  Read tiles or scanlines automatically.  Strides set to
AutoStride imply 'contiguous' data (i.e. xstride ==
spec.nchannels*ParamBaseTypeSize(format),
ystride==xstride*spec.width, zstride=ystride*spec.height).
\apiend


\apiitem{bool read_image (float *data)}
Simple read_image reads to contiguous float pixels.
\apiend

\apiitem{bool read_native_scanline (int y, int z, void *data)}
read_native_scanline is just like read_scanline, except that it
keeps the data in the native format of the disk file and always
read into contiguous memory (no strides).  It's up to the user to
have enough space allocated and know what to do with the data.
IT IS EXPECTED THAT EACH FORMAT PLUGIN WILL OVERRIDE THIS METHOD.
\apiend


\apiitem{bool read_native_tile (int x, int y, int z, void *data)}
read_native_tile is just like read_tile, except that it
keeps the data in the native format of the disk file and always
read into contiguous memory (no strides).  It's up to the user to
have enough space allocated and know what to do with the data.
IT IS EXPECTED THAT EACH FORMAT PLUGIN WILL OVERRIDE THIS METHOD
IF IT SUPPORTS TILED IMAGES.
\apiend

\apiitem{int send_to_input (const char *format, ...)}
General message passing between client and image input server
\apiend

\apiitem{int send_to_client (const char *format, ...)}
General message passing between client and image input server
\apiend

\apiitem{std::string error_message () const}
Return the current error string describing what went wrong if
any of the public methods returned 'false' indicating an error.
(Hopefully the implementation plugin called error() with a
helpful error message.)
\apiend

\end{comment}

\apiitem{}
\apiend

\apiitem{}
\apiend

\apiitem{}
\apiend

\apiitem{}
\apiend

\apiitem{}
\apiend

\apiitem{}
\apiend

\apiitem{}
\apiend

\apiitem{}
\apiend

\apiitem{}
\apiend

\apiitem{}
\apiend

\apiitem{}
\apiend

\apiitem{}
\apiend

\apiitem{}
\apiend

\apiitem{}
\apiend

\apiitem{}
\apiend


\index{Image I/O API|)}

\chapwidthend
